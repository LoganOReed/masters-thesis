\chapter{Preliminaries}
\thispagestyle{myheadings}

\section{Measure Theory}
\begin{definition}
  A Measure space is a pair $(X,A)$ where $X$ is a non-empty set and $A$ is a $\sigma$-algebra of subsets of $X$.
  That is, $A$ satisfies the following
  \begin{enumerate}
    \item $\emptyset \in A$
    \item For countably many $A_{j} \in A$, $\cup A_{j} \in A$ 
    \item If $B \in A$, then $X - B \in A$.
  \end{enumerate}
\end{definition}
Elements in $A$ are called measurable sets.
A measure $\mu$ on $(X,A)$ is a non-negative function $ \mu : A \to [0,\infty] $ such that $\mu(\emptyset) = 0$ and $\mu(\cup A_{j}) = \Sigma \mu(A_{j})$ for any countably many, mutually disjoint $A_{j} \in A$.
$\mu$ is said to be finite if $\mu(X) < \infty$, and $\mu$ is $\sigma$-finite if $X$ is a countable union of sets in $A$ with finite measures.
A property is said to hold almost everywhere on a set $A$ if it holds on $A$ except a subset with zero measure.
A function $ f : X \to [-\infty, \infty] $ is measurable if $ \left\{ x \in X : f(x) < \alpha \right\}$ is measurable for all $\alpha \in \mathbb{R}^{} $.
A simple function  is a function of the form
\[
f = \sum_{j=1}^m a_{j}\chi_{A_{j}} 
,\] 
where $\chi_{S}$ is the characteristic function on the set $S$, $a_{j} \in \mathbb{R}^{} $, and $A_{j} \in A$.
\begin{theorem}[Simple Function Approximation Theorem]
  Let $ f : X \to [-\infty, \infty] $ be a measurable function.
  If $f$ is non-negative, then there exists an increasing sequence of simple functions $\phi_{j}$ such that $0 \leq \phi_{j} \leq f$ and $\lim_{j \to \infty} \phi_{j}(x) = f(x)$.
  If $f$ is bounded, then there exists a sequence of simple functions $\phi_{j}$ such that $\phi_{j} \to f$ uniformly on $X$.
\end{theorem}
We will assume our measure spaces $(X,A,\mu)$ are complete.
That is, if $B \subset N$ and $\mu(N) = 0$ then $B \in A$.
The integration with respect to $\mu$ is defined in the following way.
We first define the integral for non-negative simple functions.
Let $\phi = \sum_{j=1}^m a_{j} \chi_{A_{j}} \geq 0$, and define 
\[
\int_{ X} \! \phi \, \mathrm{d}\mu = \sum_{j=1}^m a_{j} \mu(A_{j})
\] 
where we use the convention that $0 \cdot \infty = 0$.
We define the integral for non-negative measurable functions as
\[
  \int_{ X} \! f \, \mathrm{d}\mu = \sup \left\{ \int_{ X} \! \phi \, \mathrm{d}\mu ; 0 \leq \phi \leq f, \phi \text{ simple}  \right\} 
.\] 
For a measurable function $ f : X \to [-\infty,\infty] $ we write $f^{+} = \max \left\{ f,0 \right\}$ and $f^{-} = \max \left\{ -f,0 \right \}$, and we define
\[
\int_{ X} \! f \, \mathrm{d}\mu = \int_{ X} \! f^{+} \, \mathrm{d}\mu - \int_{ X} \! f^{-} \, \mathrm{d}\mu  
\] 
given at least one of the integrals on the right hand side is finite.
When $f$ is complex-valued we define the integral by integrating the real and complex parts seperately.
When $\int_{ X} \! |f| \, \mathrm{d}\mu < \infty $ we say $f$ is integrable.
\begin{definition}
  If $0 < p < \infty$ and if $f$ is a complex measurable function on $X$, define
  \[
    ||f||_{p} = \left\{ \int_{ X} \! |f|^{p} \, \mathrm{d}\mu  \right\}^{\frac{1}{p}}
  \] 
  and let $L^{p}(\mu)$ consist of all $f$ for which $||f||_{p} < \infty$.
\end{definition}
For $ f : X \to \mathbb{C} $, the support of $f$ is defined as supp$(f) = \overline{\left\{ x \in X; f(x) \not = 0 \right\}} $.
Denote by $C_{c}(X)$ the family of continuous functions on $X$ with compact support and by $C_{0}(X)$ the family of continuous functions that vanish at infinity.
We refer the interested readers to the book of Royden\cite{royden}.

\section{Functional Analysis}
\begin{definition}
A complex linear space $\mathbb{H}$ is called a normed linear space if there exists a map $|| \cdot || : \mathbb{H} \to \mathbb{R}^{+} $ such that for any $x, y \in \mathbb{H}$ and $ \lambda \in \mathbb{C}$,
\begin{enumerate}
  \item $|| \lambda x || = |\lambda| || x ||$
  \item $|| x + y || \leq || x || + || y ||  $
  \item $|| x || \geq 0$, and $|| x || = 0$ if and only if $x = 0$
\end{enumerate}
We call this map a norm.
\end{definition}
\begin{definition}
  A Banach space $X$ is a complete, normed linear space.
\end{definition}
\begin{definition}
  We say $X$ is seperable if $X$ contains a countable dense subset.
\end{definition}
\begin{definition}
  A complex linear space $\mathbb{H}$ is called an inner product space with inner product $ \langle \cdot, \cdot \rangle  : \mathbb{H} \times \mathbb{H} \to \mathbb{C} $ if for any $x,y,z \in \mathbb{H}$ and $\lambda \in \mathbb{C}$,
  \begin{enumerate}
    \item $ \langle \lambda x, y \rangle = \lambda \langle x, y \rangle  $
    \item $ \langle x,y \rangle = \overline{ \langle y,x \rangle }$
    \item $ \langle x + y, z \rangle = \langle x,z \rangle + \langle y,z \rangle $
    \item $ \langle x,x \rangle \geq 0$, and $ \langle x,x \rangle = 0 $ if and only if $x = 0$.
  \end{enumerate}
\end{definition}
For an inner product $ \left< , \right>$, the associated norm is $|| u || := \left< u,u \right>^{\frac{1}{2}}$ for $u \in \mathbb{H}$.
One could verify the triangle inequality in the definition of a norm via the Cauchy-Schwarz inequality.
We say that two elements $u,v \in \mathbb{H}$ are orthogonal if $\left< u,v \right> = 0$.
A countable basis $\{w_{k}\}_{k=1}^\infty \subset \mathbb{H}$ is orthonormal if the elements are pairwize orthogonal and the norm of each element is one.
\begin{definition}
  A Hilbert space $\mathbb{H}$ is a Banach space endowed with an inner product which generates the norm.
\end{definition}
For the remainder of this paper all Hilbert spaces will be assumed to be seperable.

Let $X,Y$ be real Banach spaces.
\begin{definition}
  A mapping $ A : X \to Y $ is a linear operator provided
  \[
  A(au + bv) = aAu + bAv
  \] 
  for all $u,v \in X$ and $a,b \in \mathbb{R}^{} $.
\end{definition}
\begin{definition}
  A linear operator $ A : X \to Y $ is bounded if 
  \[
  || A || := \sup \left\{ || Au ||; || u || \leq 1  \right\} \leq \infty
  .\] 
\end{definition}
\begin{definition}
  Let $T \in L(\mathbb{H})$. $T$ is symmetric if $T \subset T^{*}$; namely, $\text{Dom}(T) \subset \text{Dom}(T^{*})$ and 
  \[
  \left<Tx,y \right> = \left<x,Ty \right>
  \] 
  for all $x,y \in \text{Dom}(T)$. T is self-adjoint if $T = T^{*}$, namely, $T$ is symmetric and $\text{Dom}(T) = \text{Dom}(T^{*})$.
\end{definition}
\begin{definition}
  A linear operator $ A : X \to Y $ is closed if whenever $u_{k} \to u$ in $X$ and $Au_{k} \to v$ in $Y$, then $Au = v$
\end{definition}
\begin{definition}
  Let $T \in L(\mathbb{H})$ be a non-negative, self-adjoint operator. The form defined by
  \[
  Q(f,g) = \left( T^{1 / 2}f,T^{1 / 2}g \right)
  \] 
  with $\text{Dom}(Q) = \text{Dom}(T^{1 / 2})$ is called the sesquilinear form associated with $T$.
\end{definition}
The following theorem gives an existence proof for a sequence of eigenvalues and eigenfunctions\cite{evans}
\begin{theorem} \label{spectrum}
  Let $\mathbb{H}$ be a separable Hilbert space of infinite dimension and $T$ a self-adjoint, compact, and positive operator. 
  Then there exists a sequence of real positive eigenvalues $(v_{n})$, $n \geq 1$ converging to zero and a sequence of eigenvectors $(x_{n})$, $n \geq 1$ defining a Hilbert basis of $\mathbb{H}$ such that for all $n$, $Tx_{n} = v_{n} x_{n}$.
\end{theorem}

Let $f \in C^{\infty}(\Omega)$.
Then using integration by parts we have 
\[
\left( \partial^{\alpha}f, \phi \right) = \left( f, (-1)^{| \alpha |} \partial^{\alpha} \phi \right)
\] 
for any $\phi \in C_{c}^{\infty}$.
We denote $L_{loc}(\Omega)$ to be the space of Lebesgue measurable functions that are integrable over a compact subset of $\Omega$.
\begin{definition}
 Let $f,g \in L_{loc}(\Omega)$. We say that $g = \partial^{\alpha} f$ in the sense of distribution if $\left( f, (-1)^{| \alpha |} \partial^{\alpha} \phi \right) = \left( g,\phi \right)$ for any $\phi \in C_{c}^{\infty}(\Omega)$.
\end{definition}
The following is primarily from Henrot\cite{henrot}.
Let $\Omega$ be a bounded open set in $\mathbb{R}^{N} $.
We denote by $L^{2}(\Omega)$ the Hilbert space of square integrable functions defined on $\Omega$ and by $H^{1}(\Omega)$ the Sobolev space of functions in $L^{2}(\Omega)$ whose partial derivatives (in the sense of distrubutions) are in $L^{2}(\Omega)$.
When $H^{1}(\Omega)$ is endowed with the scalar product
\[
  \left( u,v \right)_{H^{1}} := \int_{ \Omega} \! u(x)v(x) \, \mathrm{d}x + \int_{ \Omega} \! \nabla u(x) \nabla v(x) \, \mathrm{d}x 
\] 
and the corresponding norm 
\[
  || u ||_{H^{1}} := \left( \int_{ \Omega} \! u(x)^{2} \, \mathrm{d}x + \int_{ \Omega} \! | \nabla u(x) |^{2} \, \mathrm{d}x  \right)^{\frac{1}{2}}
\] 
$H^{1}(\Omega)$ becomes a Hilbert space.
In the case of Dirichlet boundary conditions, we will use the subspace $H_{0}^{1}(\Omega)$ which is defined as the closure of $C^{\infty}$ functions compactly supported in $\Omega$ for the norm $|| \cdot ||_{H^{1}}$.
This is also a Hilbert space.

We will occasionally need to work with spaces $L^{p}, p \geq 1$, instead of $L^{2}$.
In this case we define the Sobolev spaces in the same way and denote them $W^{1,p}(\Omega)$ and $W^{1,p}_{0}(\Omega)$ respectively.
These are Banach spaces.
By definition $H_{0}^{1}(\Omega)$ and $H^{1}(\Omega)$ are continuously embedded in $L^{2}(\Omega)$, but we will eventually need a compact embedding.
The following theorem (see Vladimir Maz’ya\cite{sobolev}) will give us just that
\begin{theorem}[Rellich] \label{rellich}
    For any bounded open set $\Omega$, the embedding $H_{0}^{1}(\Omega) \to L^{2}(\Omega)$ is compact.
    Also, if $\Omega$ is a bounded open set with Lipschitz boundary, the embedding $H^{1} \to L^{2}(\Omega)$ is compact.
\end{theorem}

\section{Partial Differential Equations}
We begin by defining the Laplacian Operator 
\[
- \Delta u := - \sum_{i=1}^N \pderiv[2]{u}{x_{i}} 
,\] 
where derivatives are to be understood in the sense of distrubutions.
The following two results will be used in the definition of the Dirichlet Laplacian.
Proofs of both theorems can be found in the Davies' texbook on Spectral Theory \cite{davies}.
\begin{theorem} \label{sesqq}
  Let $T \in L(\mathbb{H})$ be non-negative and self-adjoint, and let $Q$ be the associated quadratic form.
  Then 
  \begin{enumerate}
    \item $f \in \text{Dom}(T)$ if and only if $f \in \text{Dom}(Q)$ and there exists a vector $g \in \mathbb{H}$ such that $Q(f,\phi) = (g,\phi)$ for all $\phi \in \text{Dom}(Q)$.
      In this case $g = Tf$.
    \item $\text{Dom}(T)$ is dense in $\text{Dom}(Q)$ in $|| \cdot ||_{1}-norm$.
  \end{enumerate}
\end{theorem}
\begin{theorem} \label{sesq}
  Let $Q$ be a non-negative sesquilinear form with dense domain $\text{Dom}(Q)$ in $\mathbb{H}$. 
  The following are equivalent
  \begin{enumerate}
    \item $Q$ is closed.
    \item The associated quadratic form $\widetilde{Q}$ is lower semi-continuous.
    \item There exists a unique non-negative, self-adjoint operator $T$ such that $\text{Dom}(T^{\frac{1}{2}}) = \text{Dom}(Q)$ and $Q(f,g) = \left( T^{\frac{1}{2}}f,T^{\frac{1}{2}}g \right)$.
  \end{enumerate}
\end{theorem}
\begin{definition}
  Let $\Omega$ be an open set in $\mathbb{R}^{N} $, and let $\nabla = (\pderiv[1]{}{x_1} , \pderiv[1]{}{x_2} , \ldots, \pderiv[1]{}{x_{N}} )$ be the gradient operator.
  Let 
  \[
  Q_{\Omega}^{D}(u,v) = \left( \nabla u, \nabla v \right), \,\, \text{Dom}(Q_{\Omega}^{D}) = W_{0}^{1}(\Omega)
  .\] 
  Since $C_{c}^{\infty}(\infty) \subset H_{0}^{1}(\Omega) \subset L^{2}(\Omega)$ and $C_{c}^{\infty}$ is dense in $L^{2}(\Omega)$, the sesquilinear form $Q_{\Omega}^{D}$ is densely defined on $L^{2}(\Omega)$.
  Also, since $W_{0}^{1}(\Omega)$ is complete, $Q_{\Omega}^{D}$ is closed.
  Then, by (\ref{sesq}), $Q_{\Omega}^{D}$ uniquely determines a densely defined self adjoint operator $ \Delta  : L^{2}(\Omega) \to L^{2}(\Omega) $ such that 
  \[
  Q_{\Omega}^{D}(u,v) = \left( \Delta^{\frac{1}{2}} u, \Delta^{\frac{1}{2}} v \right), \,\, \text{Dom}(\Delta^{\frac{1}{2}}) = \text{Dom}(Q^{D}) = W_{0}^{1}(\Omega)
  .\] 
  The operator $\Delta $ thus defined is called the Dirichlet Laplacian, and we will denote it by $\Delta_{\Omega}^{D}$.
\end{definition}
  In order for this definition to be useful, we will show this definition is equivalent to the standard definition of the Dirichlet Laplacian we used in the introduction.
  By (\ref{sesqq}), $f \in \text{Dom}(\Delta^{D})$ if and only if $f \in W_{0}^{1}(\Omega)$ and there exists some $g \in L^{2}(\Omega)$ such that 
  \[
  \left( g,\phi \right) = Q^{D}(f,\phi) = \left( \nabla f, \nabla \phi \right) = \left( - \nabla^{2} f, \phi \right)
  \] 
  for all $\phi \in C_{0}^{\infty}(\Omega)$.
  Thus $-\nabla^{2} f = g \in L^{2}(\Omega)$ in the sense of distribution.
  Therefore we have
  \[
    \Delta^{D} = -\nabla^{2} = - \sum_{j=1}^{N} \pderiv[2]{}{x_{j}}, \, \text{Dom}(\Delta^{D}) = W_{0}^{1}(\Omega) \cap \left\{ f \in L^{2}(\Omega) : \nabla^{2}f \in L^{2}(\Omega) \right\}
  .\] 
  Since $f \in W_{0}^{1}$ implies the trace of $f = 0$ on the boundary of $\Omega$, we have shown this definition is consistent with the classical definition.

  Next, lets state a general property of the Laplacian that will be used constantly.
\begin{theorem}
  The Laplacian is invariant under orthogonal transformations.
\end{theorem}
\begin{proof}
  Let $f \in C_{c}^2(\mathbb{R}^{N} )$ and let $A$ be an orthogonal $n \times n$ matrix over $\mathbb{R}^{} $.
  Also, let $x = (x_1,x_2,\ldots,x_{N})$.
  Since $A$ is orthogonal, $\sum_{j=1}^N a_{ij}a_{kj} = \delta_{ik}$ where $\delta_{ik}$ is the Kronecker Delta function.
  So we have
  \[
    (f \circ A)(x) = f \left( \sum_{i=1}^{N} a_{1i}x_{i}, \ldots , \sum_{i=1}^{N} a_{di}x_{i} \right )
  .\] 
  Take $z_{i} = g_{i}(x_1,x_2,\ldots,x_{N}) = \sum_{k=1}^{N} a_{ik}x_{k}$.
  From a direct application of the chain rule we obtain

  \[
    \frac{d}{dx_j} (f\circ A)(x) = \sum_{k=1}^N a_{kj} \cdot (\partial_k f)( \sum_{i=1}^N a_{1i} x_i, \dots, \sum_{i=1}^N a_{di} x_i ) 
  .\] 
  Further, by taking $\partial_{k} f$ in place of $f$, we obtain
  \[
  \frac{d^2}{dx_j^2} (f\circ A)(x) = \sum_{k=1}^N a_{kj} \sum_{\ell=1}^N a_{\ell j} (\partial_\ell \partial_k f)( \sum_{i=1}^N a_{1i} x_i, \dots, \sum_{i=1}^N a_{di} x_i )
  .\] 
  With all of these pieces in place, we have the following
\begin{align*}
  \Delta (f\circ A)(x) &= \sum_{j=1}^N \frac{d^2}{dx_j^2} (f\circ A)(x) \\
  &= \sum_{j=1}^N \sum_{k=1}^N a_{kj} \sum_{\ell=1}^N a_{\ell j} (\partial_\ell \partial_k f)( \sum_{i=1}^N a_{1i} x_i, \dots, \sum_{i=1}^N a_{di} x_i ) \\
  &= \sum_{k,\ell=1}^N \left( \sum_{j=1}^N a_{kj} a_{\ell j} \right) (\partial_\ell \partial_k f)( \sum_{i=1}^N a_{1i} x_i, \dots, \sum_{i=1}^N a_{di} x_i ) \\
  &= \sum_{k,\ell=1}^N \delta_{k,\ell} (\partial_\ell \partial_k f)( \sum_{i=1}^N a_{1i} x_i, \dots, \sum_{i=1}^N a_{di} x_i ) \\
  &= \sum_{k=1}^N (\partial_k^2 f)( \sum_{i=1}^N a_{1i} x_i, \dots, \sum_{i=1}^N a_{di} x_i ) \\
  &= (\Delta f) ( \sum_{i=1}^N a_{1i} x_i, \dots, \sum_{i=1}^N a_{di} x_i ) \\
  &= (\Delta f)(Ax) \\
  &= ((\Delta f)\circ A)(x).
\end{align*}
  Hence $\Delta (f\circ A)(x) = ((\Delta f)\circ A)(x)$ and so the Laplacian is invariant under orthogonal transformations.
\end{proof}
We will use this result thoughout the rest of the proofs without explicit reference, especially when using translations and rotations.

\section{Tools}
In this section we will introduce two tools that allow us to construct minimizing sequences of a domain, which are immensely useful whenever we are able to apply them.
The first one, called Schwarz Rearrangement, acts on domains in $\mathbb{R}^{N}$ and essentially is the process of taking the original domain and squishing it into a ball while maintaining the same volume.
This process causes certain properties of the domain to decrease, and since the volume is constant we can show there is a shape that minimizes the decreasing property even when we constrain the volume.
We will use this line of thought to prove the Faber-Krahn inequality (Theorem \ref{fk}).
Our other tool, Steiner Symmetrization, can be thought of as a more delicate way of obtaining the same end result.
While Schwarz Rearrangement more or less just pushes everything together, the idea behind Steiner Symmetrization is that we can find a specific line where the domain must be symmetric under it's reflection.
Moreover, by continuing to find these specific lines of symmetry we can deform the domain into a ball.
This is the tool Pólya used to prove the Pólya-Szegő conjecture for $N=3,4$.
For a more complete view of what these tools allow us to prove, we recommend the following texts \cite{polya}\cite{isoperimetric}\cite{henrot2}.
\begin{definition}[Schwarz Rearrangement] \label{schwarz}
  For any measurable set $\Omega$ in $\mathbb{R}^{N} $, we denote by $\Omega^{*}$ the ball of same volume as $\Omega$.
  If $u$ is a non-negative measurable function defined on a measurable set $\Omega$ and vanishing on its boundary $\partial \Omega$,
  we denote by $\Omega(c) = \{ x \in \Omega \,|\, u(x) \geq c \}$ its level sets.
  The Schwarz rearrangement of $u$ is the function $u^{*}$ defined on $\Omega^{*}$ by 
  \[
    u^{*}(x) = \sup\{c / x \in \Omega(c)^{*}\}
  .\] 
  
\end{definition}
Without loss of generality, we fix the hyperplane of symmetry to be $x_{N} = 0$.
Let $N \geq 2$ and $\Omega \subset \mathbb{R}^{N} $ be a measurable set.
We denote by $\Omega'$ the projection of $\Omega$ on $\mathbb{R}^{N-1} $, and for every $x' \in \mathbb{R}^{N-1} $ we denote by $\Omega(x')$ the projection of $\Omega$ with $ \left\{ x' \right\} \times \mathbb{R}^{}  $.
\begin{definition}[Steiner Symmetrization] \label{steiner}
Let $\Omega \subset \mathbb{R}^{N} $ be measurable.
Then the set 
\[
\Omega^{*} := \left\{ x = (x', x_{N}) : - \frac{1}{2} |\Omega(x')| < x_{N} < \frac{1}{2} |\Omega(x')|, x' \in \Omega' \right\} 
\] 
is the Steiner symmetrization of $\Omega$ with respect to the hyperplane $x_{N}$ = 0.
\end{definition}
\begin{theorem} \label {fk1}
 Let $\Omega$ be a measurable set and $u$ be a non-negative measurable function defined on $\Omega$ and vanishing on its boundary $\partial  \Omega$. 
 Let $\phi$ be any measurable function defined on $\mathbb{R}^{+} $ with values in $\mathbb{R}^{} $, then
 \[
   \int_\Omega \! \phi(u(x)) \, \mathrm{d}x = \int_{\Omega^*} \! \phi(u^*(x)) \, \mathrm{d}x  
 .\] 
\end{theorem}
\begin{theorem}[Pólya's Inequality] \label{fk2}
  Let $\Omega$ be an open set and $u$ a non-negative function belonging to the Sobolev space $H_{0}^{1}(\Omega)$.
  Then $u^{*} \in H_{0}^{1}(\Omega^{*})$ and 
  \[
    \int_\Omega \! | \nabla u(x)  |^2  \, \mathrm{d}x \geq \int_{\Omega^{*}} \! | \nabla u^{*}(x) |^2 \, \mathrm{d}x 
  .\] 
\end{theorem}
