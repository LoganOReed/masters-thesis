\documentclass[12pt]{report}

\usepackage[a4paper,bindingoffset=0.2in,%
            left=1.5in,right=1in,top=1in,bottom=1in,%
            footskip=.25in]{geometry}


\usepackage[utf8]{inputenc}
\usepackage[english]{babel}
\pagenumbering{roman}
\usepackage{graphicx} 
\usepackage{biblatex}
\usepackage{csquotes} % Biblatex throws warning if not included
\usepackage{amsmath}
\usepackage{setspace} 
\usepackage{amssymb} 
\usepackage{esint} 
\usepackage{cool} % A ridiculous number of macros for complex symbols

% specific external file paths
\addbibresource{refs.bib}
\graphicspath{ {./graphics/} }



\newtheorem{theorem}{Theorem}[section]
%\def\thetheorem{\unskip}
\newtheorem{proposition}[theorem]{Proposition}
%\def\theproposition{\unskip}
\newtheorem{conjecture}[theorem]{Conjecture}
\def\theconjecture{\unskip}
\newtheorem{corollary}[theorem]{Corollary}
\newtheorem{lemma}[theorem]{Lemma}
\newtheorem{observation}[theorem]{Observation}
%\def\thelemma{\unskip}
\newtheorem{definition}{Definition}
\numberwithin{definition}{section}
%\def\thedefinition{\unskip}
\newtheorem{remark}{Remark}
\def\theremark{\unskip}
\newtheorem{question}{Question}
\def\thequestion{\unskip}
\newtheorem{example}{Example}
\def\theexample{\unskip}
\newtheorem{problem}{Problem}
\newtheorem{exercise}[theorem]{Exercise}

\begin{document}

  \doublespacing
  
\begin{titlepage}

   \begin{center}
      

      TITLE TO BE DECIDED AT A FUTURE DATE

            
     
       By
              


      LOGAN REED

     
            
        A thesis submitted to the \\
       Graduate School–Camden\\
       Rutgers, The State University of New Jersey\\
       In partial fulfullment of the requirements\\
       For the degree of Master of Science\\
       Graduate Program in Mathematical Sciences \\
       Written under the direction of \\
       Siqi Fu\\
       And approved by \\
       \noindent\rule{4cm}{0.4pt}\\
       Dr. One\\
      \noindent\rule{4cm}{0.4pt}\\
       Dr. Two\\
           \noindent\rule{4cm}{0.4pt}\\
       Dr. Three\\
                \noindent\rule{4cm}{0.4pt}\\
       Dr. Four\\
         \vspace{0.8cm}
       Camden, New Jersey\\
       May 2023
            
       \vspace{0.8cm}
     
  
        
            
   \end{center}
   
\end{titlepage}




\break
TEST PAGE

  \begin{center}

\break
  THESIS ABSTRACT\\
 
    TODO SOMETHING ABOUT THE ABSTRACT THAT IS ABOUT THIS LONG OR SO \\
         by LOGAN REED\\
     
     Thesis Director: \\
     Siqi Fu
  \end {center}

  The topic that I chose to explore for this thesis is a study of the eigenvalues of the Dirichlet Laplacian on a two dimensional domain and \ldots
  
	% The topic that I chose to explore for this thesis is a study of the eigenvalues of the Dirichlet Laplacian on a two dimensional domain and the properties that arise as a direct consequence to them. The eigenvalues of a given domain produce so many surprising insights that simply the study of a triangular domain has many directions in which one could explore. What I love about this topic is that throughout my study, a resource from 1966 could take me to a resource from the 1700's which could lead me all the way back to 2013. It is a topic rich with exploration, both new and old which really gave me a good picture of what modern research in pure mathematics can look like.
	% 
	% The goal with this thesis was simple; understand a few of the implications brought up by the famous question of Mark Kac, ``Can one hear the shape of a drum?" and then possibly, with enough effort, make some type of original contribution to the topic. 
	% 
	% In my study of these domains, I was inevitably brought back to the isoperimetric inequality leading me to a process called Steiner Symmetrization. This process was introduced to me by \cite{Polya}.  We were able to use some basic trigonometry to fill in the detail in the treatise of George Polya and come up with simplified proofs of special cases of such symmetrization. 
	%  
\break



\begin{center}
Acknowledgment
\end{center}
    I would be remiss if I did not take a moment to express appreciation for all of the people who have helped me through this process.
    % I would be remiss if I did not take a moment to express appreciation for all of the people who have helped me through this process. 
    % 
    % Dr. Siqi Fu is a person I must thank not only for supervising my research throughout the writing of this thesis, but also helping me make the decision all those years ago to pursue both bachelor and masters degrees in mathematics and later encouraging me to continue my education in pursuit of a PhD as well. Without his guidance and encouragement at multiple steps throughout my academic career, my confidence to pursue a path of learning mathematics may have wavered or worse, not even have began. 
    % 
    % As a person that has spent my entire secondary educational career at Rutgers University–Camden, I also must thank the rest of the Department of Mathematics. At a smaller school like Rutgers I was given the unique opportunity to develop closer relationships with my professors. Because of this, at every step along the way, through each class, conversation and piece of advice, the professors in the math department at Rutgers each hold an integral role in my development as a student of mathematics. Through their mathematical and pedagogical prowess, I feel as though the Rutgers professors have prepared me to continue my education and hopefully someday make a real contribution to the field of Mathematics. For this education and guidance I am truly grateful. 
    % 
    % Lastly, I would like to thank all of my family and friends. Without the love and support of my parents, I would not have been able to pursue all of the opportunities I have been afforded at Rutgers. I would also like to thank Rachel and Dixon for encouraging me at every step along the way thus far (and spell checking this document for me). 
    





	
\break

\tableofcontents


\newcommand{\comment}[1]{}


\break


\pagenumbering{arabic}
\pagestyle{myheadings}

\section*{ Introduction }

  
This is the intro.
	

\break









\chapter{Background}

\break
\section {Physical Motivation}

% Page 66 in Evans PDE

Consider a homogeneous elastic drumhead, or membrane, stretched over a rigid frame.
We will represent the frame as a domain $\Omega \subset \mathbb{R}^{2}$.
Take the function $u(x,y,t)$ to be the vertical displacement of the membrane from its resting position.
Then for any disk $D \subset \Omega$, Newton's second law of motion states that

$$ \int_{\partial D} T \frac{\partial u}{\partial \textbf{n}} \, dS = \int_{ D} \rho u_{tt} \, dA $$

where T is the constant tension, $\rho$ is the density constant, and $\textbf{n}$ is the outward normal of the boundary.
By the divergence theorem, we have

$$ \int_{D} T \Delta u \, dA = \int_{D} \rho u_{tt} \, dA $$

where $\Delta$ is the Laplace operator. From this we can get the wave equation on $\Omega$

$$ u_{tt} = c^{2} \Delta u $$

where we define u to be 0 on the boundary and where $c = \sqrt{T / \rho}$.
We can solve this wave equation using $u(x,y,t) = T(t)V(x,y)$ which gives us

$$ \frac{T''}{c^{2}T} = \frac{\Delta V}{V} = - \lambda $$

and finally we have reduced our problem to the Dirichlet Laplacian

$$ \Delta V = - \lambda V $$

where $V$ on the boundary is zero.

NOTE: The best reference I could find is Logan's Applied Partial Differential Equations.
I could also use Ryans paper

In the next section, we will start from the Dirichlet Laplacian and introduce the conjectures in a formal setting.

\break

\section {Polya-Szego's Conjecture}


\begin{enumerate}
  \item Introduce Rigorous Definitions from 1.1.2 Henrot
  \item Dirichlet Laplacian eigenvalues prereqs
  \item Faber Krahn
  \item Polya-Szego Conjecture \cite{isoperimetric}
\end{enumerate}

\break

\section{Known Results}

\begin{enumerate}
  \item All Explicit Cases
  \item Tools for n=3 and n=4
\end{enumerate}

\break

\section {Numerical Analysis Tools}

\chapter {Hearing the shape of a triangle}
\break
\section{Section One}


\printbibliography

% \begin{thebibliography}{100}
%  \singlespacing
%  \bibitem{Kac} Kac, M., ``Can One Hear the Shape of a Drum?," \emph{The American Mathematical Monthly}, Vol.73, No 4, April 1966.
%  
%  \bibitem{Polya} Polya, G., Szego, ``Isoperimetric inequalities in mathematical physics," \emph{Annals of mathematical studies}, Princeton University Press, 1951
%
%  \bibitem{Henrot} Henrot, A. , “Extremum problems for eigenvalues of elliptic operators”, Frontiers in mathematics, Birkhuser Verlag, Basel, 2006.
% \end{thebibliography}



 
 
 
\end{document}
