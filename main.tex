\documentclass[12pt]{report}

\usepackage[a4paper,bindingoffset=0.2in,%
            left=1.5in,right=1in,top=1in,bottom=1in,%
            footskip=.25in]{geometry}


\usepackage[utf8]{inputenc}
\usepackage[english]{babel}
\pagenumbering{roman}
\usepackage{graphicx} 
\usepackage{biblatex}
\usepackage{csquotes} % Biblatex throws warning if not included
\usepackage{amsmath}
\usepackage{amsthm}
\usepackage{setspace} 
\usepackage{amssymb} 
\usepackage{esint} 
\usepackage{cool} % A ridiculous number of macros for complex symbols

% specific external file paths
\addbibresource{refs.bib}
\graphicspath{ {./graphics/} }



\newtheorem{theorem}{Theorem}[section]
%\def\thetheorem{\unskip}
\newtheorem{proposition}[theorem]{Proposition}
%\def\theproposition{\unskip}
\newtheorem{conjecture}[theorem]{Conjecture}
\def\theconjecture{\unskip}
\newtheorem{corollary}[theorem]{Corollary}
\newtheorem{lemma}[theorem]{Lemma}
\newtheorem{observation}[theorem]{Observation}
%\def\thelemma{\unskip}
\newtheorem{definition}{Definition}
\numberwithin{definition}{section}
%\def\thedefinition{\unskip}
\newtheorem{remark}{Remark}
\def\theremark{\unskip}
\newtheorem{question}{Question}
\def\thequestion{\unskip}
\newtheorem{example}{Example}
\def\theexample{\unskip}
\newtheorem{problem}{Problem}
\newtheorem{exercise}[theorem]{Exercise}


% Various rules for hiding environments

\newif\ifhideproofs
% \hideproofstrue %uncomment to hide proofs

\ifhideproofs
\usepackage{environ}
\NewEnviron{hide}{}
\let\proof\hide
\let\endproof\endhide
\fi

% notes environ

\newif\ifhidetodos

% \hidetodostrue %uncomment to disable todos package

\ifhidetodos
\usepackage[disable]{todonotes}
\else
\usepackage{todonotes}
\fi


\begin{document}

  \doublespacing
  
\begin{titlepage}

   \begin{center}
      

      The Pólya–Szegő Conjecture on Polygons: A Numerical Approach

            
     
       By
              


      LOGAN REED

     
            
        A thesis submitted to the \\
       Graduate School–Camden\\
       Rutgers, The State University of New Jersey\\
       In partial fulfullment of the requirements\\
       For the degree of Master of Science\\
       Graduate Program in Mathematical Sciences \\
       Written under the direction of \\
       Siqi Fu\\
       And approved by \\
       \noindent\rule{4cm}{0.4pt}\\
       Dr. One\\
      \noindent\rule{4cm}{0.4pt}\\
       Dr. Two\\
           \noindent\rule{4cm}{0.4pt}\\
       Dr. Three\\
                \noindent\rule{4cm}{0.4pt}\\
       Dr. Four\\
         \vspace{0.8cm}
       Camden, New Jersey\\
       May 2023
            
       \vspace{0.8cm}
     
  
        
            
   \end{center}
   
\end{titlepage}




\break
TEST PAGE

  \begin{center}

\break
  THESIS ABSTRACT\\
 
  \todo[inline]{Write Abstract}
    TODO SOMETHING ABOUT THE ABSTRACT THAT IS ABOUT THIS LONG OR SO \\
         by LOGAN REED\\
     
     Thesis Director: \\
     Siqi Fu
  \end {center}

  The topic that I chose to explore for this thesis is a study of the eigenvalues of the Dirichlet Laplacian on a two dimensional domain and \ldots
  
	% The topic that I chose to explore for this thesis is a study of the eigenvalues of the Dirichlet Laplacian on a two dimensional domain and the properties that arise as a direct consequence to them. The eigenvalues of a given domain produce so many surprising insights that simply the study of a triangular domain has many directions in which one could explore. What I love about this topic is that throughout my study, a resource from 1966 could take me to a resource from the 1700's which could lead me all the way back to 2013. It is a topic rich with exploration, both new and old which really gave me a good picture of what modern research in pure mathematics can look like.
	% 
	% The goal with this thesis was simple; understand a few of the implications brought up by the famous question of Mark Kac, ``Can one hear the shape of a drum?" and then possibly, with enough effort, make some type of original contribution to the topic. 
	% 
	% In my study of these domains, I was inevitably brought back to the isoperimetric inequality leading me to a process called Steiner Symmetrization. This process was introduced to me by \cite{Polya}.  We were able to use some basic trigonometry to fill in the detail in the treatise of George Polya and come up with simplified proofs of special cases of such symmetrization. 
	%  
\break



\begin{center}
Acknowledgment
\end{center}
    I would be remiss if I did not take a moment to express appreciation for all of the people who have helped me through this process.
    % I would be remiss if I did not take a moment to express appreciation for all of the people who have helped me through this process. 
    % 
    % Dr. Siqi Fu is a person I must thank not only for supervising my research throughout the writing of this thesis, but also helping me make the decision all those years ago to pursue both bachelor and masters degrees in mathematics and later encouraging me to continue my education in pursuit of a PhD as well. Without his guidance and encouragement at multiple steps throughout my academic career, my confidence to pursue a path of learning mathematics may have wavered or worse, not even have began. 
    % 
    % As a person that has spent my entire secondary educational career at Rutgers University–Camden, I also must thank the rest of the Department of Mathematics. At a smaller school like Rutgers I was given the unique opportunity to develop closer relationships with my professors. Because of this, at every step along the way, through each class, conversation and piece of advice, the professors in the math department at Rutgers each hold an integral role in my development as a student of mathematics. Through their mathematical and pedagogical prowess, I feel as though the Rutgers professors have prepared me to continue my education and hopefully someday make a real contribution to the field of Mathematics. For this education and guidance I am truly grateful. 
    % 
    % Lastly, I would like to thank all of my family and friends. Without the love and support of my parents, I would not have been able to pursue all of the opportunities I have been afforded at Rutgers. I would also like to thank Rachel and Dixon for encouraging me at every step along the way thus far (and spell checking this document for me). 
    





	
\break

\tableofcontents


\newcommand{\comment}[1]{}


\break


\pagenumbering{arabic}
\pagestyle{myheadings}



\chapter{Introduction}

\break
\section {Physical Motivation}


Physical drums consist of a rigid shell with a membrane which produces sound when hit.
A similar thing can be "created" in a pure mathematical setting by studing specific partial differential equations over a closed region.
Specifically, the frequencies of the drum membrane corresponds to the eigenvalues of the Dirichlet Laplacian.
This construction makes it possible to "hear" drums where the shape of the drumhead is any closed and simple curve.
To do this, we begin with the shape of our drumhead, which is a region of the real plane bounded by piecewise smooth curves.
Since we wish to emulate the physical properties of a drum, we want to define some system that models the vibration of the drum membrane which produces the sound.
This is done using the wave equation over our boundary, with the boundary condition that the function is 0 on the boundary.
The Dirichlet Laplacian allows us to model the physical vibration of the drum, and we can calculate its eigenvalues to find the fundamental frequency and overtones.
Thus, by modeling the vibration of the drumhead using the Dirichlet Laplacian and studying its properties, we can produce "sound" via the eigenvalues.

In 1877, Lord Rayleigh conjectured the following \cite{rayleigh}

\begin{quote}
If the area of a membrane be given, there must evidently be some
form of boundary for which the pitch (of the principal tone) is the
gravest possible, and this form can be no other than the circle.
\end{quote}

In 1923, Faber published a proof which was followed by an independent proof by Krahn in 1925 \cite{krahn}.
\begin{theorem}[Faber-Krahn]
 Let $c$ be a positive number and $B$ the ball with volume $c$. Then,
 \[
   \lambda_{1}(B) = \min \left\{ \lambda_{1}(\Omega), \Omega \text{ open subset of } \mathbb{R}^{N}, |\Omega| = c \right\} 
 .\] 
\end{theorem}
From the Faber-Krahn inequality, we know that for any drumhead with a given area, the circle is the one with the lowest tone.
In 1951, Polya and Szego conjectured that a similar statement holds for drumheads with a polygonal shape \cite{henrot}.
This conjecture has been shown to be true for 3 and 4 sided polygons, but remains unproven for any other number of sides.

There are two main hurdles that are halting progress on this conjecture.
The first is that the tools that were used to prove both Lord Rayleighs conjecture as well as the small cases for the Polya-Szego conjecture are not available when the number of sides is greater than four.
The main tool that is used is called Steiner Symmetrization, and when there are more than four sides this symmetrization method creates additional sides at each step.

The purpose of this paper is to show a specific method for running numerical approximations to suggest that this conjecture is indeed true.
This is done using a method based on fundamental solutions \cite{code}.
Specifically, we consider all functions that satisfy the Laplace Equation and then solve for the linear coefficients using the boundary conditions.
Once this is done we use gradient descent to find the polygon with the minimum first eigenvalue, which is equivalent to the first fundamental tone of the drum.



Consider a homogeneous elastic drumhead, or membrane, stretched over a rigid frame.
We will represent the frame as a domain $\Omega \subset \mathbb{R}^{2}$.
Take the function $u(x,y,t)$ to be the vertical displacement of the membrane from its resting position.
Then for any disk $D \subset \Omega$, Newton's second law of motion states that

$$ \int_{\partial D} T \frac{\partial u}{\partial \textbf{n}} \, dS = \int_{ D} \rho u_{tt} \, dA $$

where T is the constant tension, $\rho$ is the density constant, and $\textbf{n}$ is the outward normal of the boundary.
By the divergence theorem, we have

$$ \int_{D} T \Delta u \, dA = \int_{D} \rho u_{tt} \, dA $$

where $\Delta$ is the Laplace operator. From this we can get the wave equation on $\Omega$

$$ u_{tt} = c^{2} \Delta u $$

where we define u to be 0 on the boundary and where $c = \sqrt{T / \rho}$.
We can solve this wave equation using $u(x,y,t) = T(t)V(x,y)$ which gives us

$$ \frac{T''}{c^{2}T} = \frac{\Delta V}{V} = - \lambda $$

and finally we have reduced our problem to the Dirichlet Laplacian

$$ \Delta V = - \lambda V $$

where $V$ on the boundary is zero.

% The best reference I could find is Logan's Applied Partial Differential Equations.

\todo[inline]{Add History and problems part}
\todo[inline]{Use ppts for structure}
\todo[inline]{mention isoperimetric and add it}


\todo[inline]{Explain Issue with proof method for cases N >= 5}


\break


\section{Examples}
\todo[inline]{Build from Henrot pg 10}

We begin with a simple one dimensional case.
Let $\Omega = (0,L)$.
Solving the differential equation

\[
  \begin{cases}
    -u'' = \lambda u &  x \in \Omega \\
    u(0) = u(L) = 0 & 
  \end{cases}
,\] 
we find that the only non-trivial solutions are 
\[
  \lambda_{n} = \frac{n^2\pi^2}{L^2},\, u_{n} = \sin \left( \frac{n\pi x}{L} \right ),\, n \geq 1
.\] 


\[
\Omega = \left\{ (x,y) : 0 < x < a, 0 < y < b, a,b \in \mathbb{R}^{}  \right\}
.\] 

\todo[inline]{Finish Rect and add Tri and maybe disk}



\chapter {Background}

\break


\section{Measure Theory}

\begin{definition}
  A Measure space is a pair $(X,A)$ where $X$ is a non-empty set and $A$ is a $\sigma$-algebra of subsets of $X$.
  That is, $A$ satisfies the following
  \begin{enumerate}
    \item $\emptyset \in A$
    \item For countably many $A_{j} \in A$, $\cup A_{j} \in A$ 
    \item If $B \in A$, then $X - B \in A$.
  \end{enumerate}
\end{definition}

Elements in $A$ are called \textit{measurable sets}.
A \textit{measure}  $\mu$ on $(X,A)$ is a non-negative function $ \mu : A \to [0,\infty] $ such that $\mu(\emptyset) = 0$ and $\mu(\cup A_{j}) = \Sigma \mu(A_{j})$ for any countably many, mutually disjoint $A_{j} \in A$.
$\mu$ is said to be  \textit{finite} if $\mu(X) < \infty$, and $\mu$ is $\sigma$-finite if $X$ is a countable union of sets in $A$ with finite measures.
A property is said to hold \textit{almost everywhere} on a set $A$ if it holds on $A$ save a subset with zero measure.

A function $ f : X \to [-\infty, \infty] $ is \textit{measurable} if $ \left\{ x \in X : f(x) < \alpha \right\}$ is measurable for all $\alpha \in \mathbb{R}^{} $.
A \textit{simple function}  is a function of the form
\[
f = \sum_{j=1}^m a_{j}\chi_{A_{j}} 
,\] 
where $\chi_{S}$ is the characteristic function on the set $S$, $a_{j} \in \mathbb{R}^{} $, and $A_{j} \in A$.

\begin{theorem}[Simple Function Approximation Theorem]
  Let $ f : X \to [-\infty, \infty] $ be a measurable function.
  If $f$ is non-negative, then there exists an increasing sequence of simple functions $\phi_{j}$ such that $0 \leq \phi_{j} \leq f$ and $\lim_{j \to \infty} \phi_{j}(x) = f(x)$.
  If $f$ is bounded, then there exists a sequence of simple functions $\phi_{j}$ such that $\phi_{j} \to f$ uniformly on $X$.
\end{theorem}

\begin{proof}
  \todo[inline]{Write Proof: Fu Notes and Evans Appendix}
\end{proof}

We will assume our measure spaces $(X,A,\mu)$ are complete.
That is, if $B \subset N$ and $\mu(N) = 0$ then $B \in A$.
The \textit{integration} with respect to $\mu$ is defined in the following way.
We first define the integral for non-negative simple functions.
Let $\phi = \sum_{j=1}^m a_{j} \chi_{A_{j}} \geq 0$, and define 

\[
\int_{ X} \! \phi \, \mathrm{d}\mu = \sum_{j=1}^m a_{j} \mu(A_{j})
\] 
where we use the convention that $0 \cdot \infty = 0$.
We define the integral for non-negative measurable functions as
\[
  \int_{ X} \! f \, \mathrm{d}\mu = \sup \left\{ \int_{ X} \! \phi \, \mathrm{d}\mu ; 0 \leq \phi \leq f, \phi \text{ simple}  \right\} 
.\] 

For a measurable function $ f : X \to [-\infty,\infty] $ we write $f^{+} = \max \left\{ f,0 \right\}$ and $f^{-} = \max \left\{ -f,0 \right \}$, and we define
\[
\int_{ X} \! f \, \mathrm{d}\mu = \int_{ X} \! f^{+} \, \mathrm{d}\mu - \int_{ X} \! f^{-} \, \mathrm{d}\mu  
\] 
given at least one of the integrals on the right hand side is finite.
When $f$ is complex-valued we define the integral by integrating the real and complex parts seperately.
When $\int_{ X} \! |f| \, \mathrm{d}\mu < \infty $ we say $f$ is \textit{integrable}.

\begin{definition}
  If $0 < p < \infty$ and if $f$ is a complex measurable function on $X$, define
  \[
    ||f||_{p} = \left\{ \int_{ X} \! |f|^{p} \, \mathrm{d}\mu  \right\}^{\frac{1}{p}}
  \] 
  and let $L^{p}(\mu)$ consist of all $f$ for which $||f||_{p} < \infty$.
\end{definition}

For $ f : X \to \mathbb{C} $, the \textit{support} of $f$ is defined as supp$(f) = \overline{\left\{ x \in X; f(x) \not = 0 \right\}} $.
Denote by $C_{c}(X)$ the family of continuous functions on $X$ with compact support and by $C_{0}(X)$ the family of continuous functions that vanish at infinity.


\break



\section{Functional Analysis}

\begin{definition}
A complex linear space $\mathbb{H}$ is called a \textit{normed linear space} if there exists a map $|| \cdot || : \mathbb{H} \to \mathbb{R}^{+} $ such that for any $x, y \in \mathbb{H}$ and $ \lambda \in \mathbb{C}$,

\begin{enumerate}
  \item $|| \lambda x || = |\lambda| || x ||$
  \item $|| x + y || \leq || x || + || y ||  $
  \item $|| x || \geq 0$, and $|| x || = 0$ if and only if $x = 0$
\end{enumerate}

We call this map a norm
\end{definition}

\begin{definition}
  A Banach space $X$ is a complete, normed linear space.
\end{definition}

\begin{definition}
  We say $X$ is seperable if $X$ contains a countable dense subset.
\end{definition}

\begin{definition}
  A complex linear space $\mathbb{H}$ is called an inner product space with inner product $ \langle \cdot, \cdot \rangle  : \mathbb{H} \times \mathbb{H} \to \mathbb{C} $ if for any $x,y,z \in \mathbb{H}$ and $\lambda \in \mathbb{C}$,
  \begin{enumerate}
    \item $ \langle \lambda x, y \rangle = \lambda \langle x, y \rangle  $
    \item $ \langle x,y \rangle = \overline{ \langle y,x \rangle }$
    \item $ \langle x + y, z \rangle = \langle x,z \rangle + \langle y,z \rangle $
    \item $ \langle x,x \rangle \geq 0$, and $ \langle x,x \rangle = 0 $ if and only if $x = 0$.
  \end{enumerate}
\end{definition}

For an inner product $ \left< , \right>$, the associated \textit{norm} is $|| u || := \left< u,u \right>^{\frac{1}{2}}$ for $u \in \mathbb{H}$.
One could verify both of the definitions of a \textit{norm} via the Cauchy-Schwarz inequality.
We say that two elements $u,v \in \mathbb{H}$ are \textit{orthogonal} if $\left< u,v \right> = 0$.
A \textit{countable basis} $\{w_{k}\}_{k=1}^\infty \subset \mathbb{H}$ is \textit{orthonormal} if the elements are pairwize orthogonal and the norm of each element is one.


\begin{definition}
  A Hilbert space $\mathbb{H}$ is a Banach space endowed with an inner product which generates the norm.
\end{definition}
For the remainder of this paper all Hilbert spaces will be assumed to be seperable.

Let $X,Y$ be real Banach spaces.
\begin{definition}
  A mapping $ A : X \to Y $ is a linear operator provided
  \[
  A(au + bv) = aAu + bAv
  \] 
  for all $u,v \in X$ and $a,b \in \mathbb{R}^{} $.
\end{definition}

\begin{definition}
  A linear operator $ A : X \to Y $ is bounded if 
  \[
  || A || := \sup \left\{ || Au ||; || u || \leq 1  \right\} \leq \infty
  .\] 
\end{definition}

\begin{definition}
  A linear operator $ A : X \to Y $ is closed if whenever $u_{k} \to u$ in $X$ and $Au_{k} \to v$ in $Y$, then $Au = v$
\end{definition}

\todo[inline]{finish once I've done the important proofs}

\break


\section{PDEs}

\begin{definition}[Laplacian]
  \[
  - \Delta u := - \sum_{i=1}^N \pderiv[2]{u}{x_{i}} 
  .\] 
\end{definition}

\begin{definition}
  The Dirichlet Laplacian is the Laplace Operator subject to Dirichlet boundary conditions.
  That is, we call $u$ a solution to the Dirichlet Laplacian if $u$ is a solution to

  \[
  \begin{cases}
    \Delta u = \lambda u & \text{in } \Omega \\
    \,\,\,\,\, u = 0  & \text{on } \partial \Omega \\
  \end{cases}
  .\] 
\end{definition}

\todo[inline]{Should I write in this level or more akin to Henrots book. e.g. do I talk about the solution of Dirichlet Laplacian as actually being the unique solution to a variational problem.
Because it is self contained there are proofs on both ends of the spectrum}


\break


\section{Tools}

\begin{definition}[Schwarz Rearrangement] \label{schwarz}
  For any measurable set $\omega$ in $\mathbb{R}^{N} $, we denote by $\omega^{*}$ the ball of same volume as $\omega$.
  If $u$ is a non-negative measurable function defined on a measurable set $\Omega$ and vanishing on its boundary $\partial \Omega$,
  we denote by $\Omega(c) = \{ x \in \Omega \,|\, u(x) \geq c \}$ its level sets.
  The Schwarz rearrangement of $u$ is the function $u^{*}$ defined on $\Omega^{*}$ by 
  \[
    u^{*}(x) = \sup\{c / x \in \Omega(c)^{*}\}
  .\] 
  
\end{definition}

Without loss of generality, we fix the hyperplane of symmetry to be $x_{N} = 0$.
Let $N \geq 2$ and $\Omega \subset \mathbb{R}^{N} $ be a measurable set.
We denote by $\Omega'$ the projection of $\Omega$ on $\mathbb{R}^{N-1} $, and for every $x' \in \mathbb{R}^{N-1} $ we denote by $\Omega(x')$ the projection of $\Omega$ with $ \left\{ x' \right\} \times \mathbb{R}^{}  $.

\begin{definition}[Steiner Symmetrization] \label{steiner}
Let $\Omega \subset \mathbb{R}^{N} $ be measurable.
Then the set 
\[
\Omega^{*} := \left\{ x = (x', x_{N}) : - \frac{1}{2} |\Omega(x')| < x_{N} < \frac{1}{2} |\Omega(x')|, x' \in \Omega' \right\} 
\] 
is the Steiner symmetrization of $\Omega$ with respect to the hyperplane $x_{N}$ = 0.
\end{definition}


\begin{theorem} \label {fk1}
 Let $\Omega$ be a measurable set and $u$ be a non-negative measurable function defined on $\Omega$ and vanishing on its boundary $\partial  \Omega$. 
 Let $\phi$ be any measurable function defined on $\mathbb{R}^{+} $ with values in $\mathbb{R}^{} $, then
 \[
   \int_\Omega \! \phi(u(x)) \, \mathrm{d}x = \int_{\Omega^*} \! \phi(u^*(x)) \, \mathrm{d}x  
 .\] 
\end{theorem}

\begin{theorem}[Pólya's Inequality] \label{fk2}
  Let $\Omega$ be an open set and $u$ a non-negative function belonging to the Sobolev space $H_{0}^{1}(\Omega)$.
  Then $u^{*} \in H_{0}^{1}(\Omega^{*})$ and 
  \[
    \int_\Omega \! | \nabla u(x)  |^2  \, \mathrm{d}x \geq \int_{\Omega^{*}} \! | \nabla u^{*}(x) |^2 \, \mathrm{d}x 
  .\] 
\end{theorem}

\break

\chapter{Eigenvalues of the Dirichlet Laplacian}
\break

\section{Definition}


\begin{definition}[Rayleigh Quotient] \label{rq}
 For an operator $L$, we define the Rayleigh quotient to be 
 \[
   R_{L}[v] := \frac{\sum_{i,j=1}^N \int_{ \Omega} \! a_{ij}(x) \pderiv[1]{v}{x_{i}} \pderiv[1]{v}{x_{j}}  \, \mathrm{d}x + \int_{ \Omega} \! a_0(x)v^2(x) \, \mathrm{d}x }{\int_{ \Omega} \! v(x)^2 \, \mathrm{d}x }
 .\] 

\end{definition}

This is used to express the first eigenvalue of the Dirichlet Laplacian in the following way

\[
  \lambda_{1}(\Omega) = \min_{v \in H_{0}^{1}(\Omega), v \not = 0 } \frac{\int_{ \Omega} \! | \nabla v(x) | ^2 \, \mathrm{d}x }{\int_{ \Omega} \! v(x)^2 \, \mathrm{d}x }
.\] 

\begin{theorem} \label{der}
    Let $\Omega$ be a bounded open set. We assume that $\lambda_{k}'(\Omega)$ is simple.
  Then, the functions $t \to \lambda_{k}(t), t \to u_{t} \in L^2(\mathbb{R}^{N} )$ are differentiable at $t = 0$ with
  \[
    \lambda_{k}'(0) := - \int_\Omega \! \mathrm{div} (|\nabla u|^2 V) \, \mathrm{d}x 
  .\] 
  If, moreover, $\Omega$ is of class $C^2$ or if $\Omega$ is convex, then
   \[
    \lambda_{k}'(0) := - \int_\Omega \! \left( \pderiv[1]{u}{n}  \right)^2 V.n \, \mathrm{d}\sigma 
  \] 
  and the derivative $u'$ of $u_{t}$ is the solution of

\[ 
  \begin{cases}
    - \Delta u' = \lambda_{k}u' + \lambda_{k}'u & \mathrm{in }  \Omega \\
    \phantom{--}u'  = - \pderiv[1]{u}{n} V.n & \mathrm{on } \partial \Omega \\
    \phantom{-}\int_\Omega \! u u' \, \mathrm{d} \sigma = 0 .
  \end{cases}
\] 

\end{theorem}


\begin{theorem}
  Each eigenvalue is real.
  Furthermore, if we repeat each eigenvalue according to its (finite) multiplicity, we have
    $\Sigma = \left\{ \lambda_{k} \right\}_{k=1}^\infty$ 
    where $\Sigma$ is the set of eigenvalues, 
    $0 < \lambda_1 \leq \lambda_2 \leq \lambda_3 \leq \ldots$,
    and $\lambda_{k} \to \infty$ as $k \to \infty$.
    
\end{theorem}

\begin{proof}
\todo[inline]{Pg 335 Evans. Ask Dr. Fu about elementary proof for first statement}
\end{proof}

\break

\section{Known Results}

\todo[inline]{Rework Sections for this chapter. Known Results isn't descriptive}

\begin{theorem}
  The Laplacian is invariant under orthogonal transformations.
\end{theorem}

\begin{proof}
  Let $f \in C_{c}^2(\mathbb{R}^{N} )$ and let $A$ be an orthogonal $n \times n$ matrix over $\mathbb{R}^{} $.
  Also, let $x = (x_1,x_2,\ldots,x_{N})$.
  Since $A$ is orthogonal, $\sum_{j=1}^N a_{ij}a_{kj} = \delta_{ik}$ where $\delta_{ik}$ is the Kronecker Delta function.
  So we have
  \[
    (f \circ A)(x) = f \left( \sum_{i=1}^{N} a_{1i}x_{i}, \ldots , \sum_{i=1}^{N} a_{di}x_{i} \right )
  .\] 
  Take $z_{i} = g_{i}(x_1,x_2,\ldots,x_{N}) = \sum_{k=1}^{N} a_{ik}x_{k}$.
  From a direct application of the chain rule we obtain

  \[
    \frac{d}{dx_j} (f\circ A)(x) = \sum_{k=1}^N a_{kj} \cdot (\partial_k f)( \sum_{i=1}^N a_{1i} x_i, \dots, \sum_{i=1}^N a_{di} x_i ) 
  .\] 
  Further, by taking $\partial_{k} f$ in place of $f$, we obtain
  \[
  \frac{d^2}{dx_j^2} (f\circ A)(x) = \sum_{k=1}^N a_{kj} \sum_{\ell=1}^N a_{\ell j} (\partial_\ell \partial_k f)( \sum_{i=1}^N a_{1i} x_i, \dots, \sum_{i=1}^N a_{di} x_i )
  .\] 

  With all of these pieces in place, we have the following

\begin{align*}
  \Delta (f\circ A)(x) &= \sum_{j=1}^N \frac{d^2}{dx_j^2} (f\circ A)(x) \\
  &= \sum_{j=1}^N \sum_{k=1}^N a_{kj} \sum_{\ell=1}^N a_{\ell j} (\partial_\ell \partial_k f)( \sum_{i=1}^N a_{1i} x_i, \dots, \sum_{i=1}^N a_{di} x_i ) \\
  &= \sum_{k,\ell=1}^N \left( \sum_{j=1}^N a_{kj} a_{\ell j} \right) (\partial_\ell \partial_k f)( \sum_{i=1}^N a_{1i} x_i, \dots, \sum_{i=1}^N a_{di} x_i ) \\
  &= \sum_{k,\ell=1}^N \delta_{k,\ell} (\partial_\ell \partial_k f)( \sum_{i=1}^N a_{1i} x_i, \dots, \sum_{i=1}^N a_{di} x_i ) \\
  &= \sum_{k=1}^N (\partial_k^2 f)( \sum_{i=1}^N a_{1i} x_i, \dots, \sum_{i=1}^N a_{di} x_i ) \\
  &= (\Delta f) ( \sum_{i=1}^N a_{1i} x_i, \dots, \sum_{i=1}^N a_{di} x_i ) \\
  &= (\Delta f)(Ax) \\
  &= ((\Delta f)\circ A)(x).
\end{align*}
  Hence $\Delta (f\circ A)(x) = ((\Delta f)\circ A)(x)$ and so $f$ is invariant under orthogonal transformations.
  
\end{proof}

We will use this result thoughout the rest of the proofs without explicit reference, especially when using translations and rotations.

Let $k > 0$ and $H_{k}$ be a homothety of origin $\alpha$ and ratio $k$.
That is, $H_{k}(x) := kx$.
For a function $u$ defined on  $\Omega$, we define the function $H_{k}u$ on $H_{k}(\Omega)$ by $H_{k}u(x) := u(x/k)$.
Since $H_{k} \circ \Delta = k^2 \Delta \circ H_{k}$, we have
\[
\lambda_{n}(H_{k}(\Omega)) = \frac{\lambda_{n}(\Omega)}{k^2}
.\] 

Using these basic properties, we can construct a correspondence between two minimization problems \cite{henrot}.

\begin{theorem} \label{eqmin}
The minimization problems $\min \left\{ \lambda_{n}(\Omega); |\Omega| = c \right\} $ as well as $\min \left\{ |\Omega|^{2/N}  \lambda_{n}(\Omega) \right\} $ are equivalent.
That is, there exists a bijective correspondence between the solutions of these two problems.
\end{theorem}

Further, as the functional $\Omega \mapsto | \Omega |^{2 / N} \lambda_{n}(\Omega) $ is invariant under homothety, we can construct the coorespondence explicitely as follows.
Every solution of $\min \left\{ \lambda_{n}(\Omega); |\Omega| = c \right\} $ is a solution of $\min \left\{ |\Omega|^{2/N}  \lambda_{n}(\Omega) \right\} $.
In the other direction, if $\Omega$ is a solution of $\min \left\{ |\Omega|^{2/N}  \lambda_{n}(\Omega) \right\} $ with volume $c'$, then for $k = \frac{c}{c'}^{1 / N}$ the homothety $H_{k}(\Omega)$ is a solution of $\min \left\{ \lambda_{n}(\Omega); |\Omega| = c \right\} $.



\begin{theorem}[Faber-Krahn] \label{fk}
 Let $c$ be a positive number and $B$ the ball with volume $c$. Then,
 \[
   \lambda_{1}(B) = \min \left\{ \lambda_{1}(\Omega), \Omega \text{ open subset of } \mathbb{R}^{N}, |\Omega| = c \right\} 
 .\] 
\end{theorem}

\begin{proof}
  This proof is a straightforward application of Schwarz rearrangement (\ref{schwarz}) \cite{henrot}.
  Let $\Omega$ be a bounded open set of measure $c$ and $\Omega^* = B$ be the ball of the same volume.
  Let $u_1$ be a en eigenfunction with eigenvalue $\lambda_{1}(\Omega)$ and $u_1^*$ its Schwarz rearrangement.
  Using \ref{fk1} we have
  \[
   \int_{\Omega^{*}} \! u_1^{*}(x)^2 \, \mathrm{d}x = \int_{ \Omega} \! u_1(x)^2 \, \mathrm{d}x
  .\] 
  Further, using \ref{fk2} we have
  \[
     \int_{\Omega^{*}} \! | \nabla u_1^{*}(x) |^2 \, \mathrm{d}x \leq \int_\Omega \! | \nabla u_1(x)  |^2  \, \mathrm{d}x
  .\] 

  Using Rayleigh quotients (\ref{rq}) we get the following

  \[
    \lambda_{1}(\Omega^{*}) \leq \frac{\int_{\Omega} \! | \nabla u_1^{*}(x) |^2 \, \mathrm{d}x }{\int_{\Omega} \! u_1^{*}(x)^2 \, \mathrm{d}x }
  .\] 

  \[
    \lambda_{1}(\Omega) = \frac{\int_{ \Omega} \! | \nabla u_1(x) | ^2 \, \mathrm{d}x }{\int_{ \Omega} \! u_1(x)^2 \, \mathrm{d}x}
  .\] 

  Using the previous two statements yields the desired results.
\end{proof}

\break

\section{Continuity}
For the purpose of the paper, we will only be considering continuity with variable domains.

\todo[inline]{I need to add essentially all of 2.3.3 in Henrots book pg 28}
\todo[inline]{Look into sources for proofs}

\break

\section{Polygons}

Note $P_{N}$ is the class of plane polygons with at most N edges.

\begin{theorem}
  Let $a > 0$ and $N \in \mathbb{N}$ be fixed.
  Then the problem
  \[
    \min \left\{ \lambda_{1}(\Omega), \Omega \in P_{N}, |\Omega| = a \right\} 
  \] 
  has a solution.
\end{theorem}

  \todo[inline]{47 henrot}
\begin{proof}
  We will use the direct method of calculus of variations.
  Let $\Omega_{n}$ be a minimizing sequence in $P_{N}$ for $\lambda_1$.
  We will begin by showing the diameter $D(\Omega_{n})$ is bounded.
  Assume, for contradiction, that this is not the case.

  \todo[inline]{I'd like to go over this section and rewrite it.}
  Then, since the area must be fixed, we can choose some length going to infinity but with a width, for example at its basis $A_{n}B_{n}$ going to zero.
  Let us now construct another minimizing sequence $\widetilde{\Omega_{n}}$ by cutting the pick at its basis.
  Let $\widetilde{\Omega_{n}}$ be the polygon we obtain by replacing our choice by the segment $A_{n}B_{n}$.
  Obviously $| \widetilde{\Omega_{n}} | \leq | \Omega_{n} | $, so if we prove that $\lambda_{1}(\widetilde{\Omega_{n}}) - \lambda_{1}(\Omega_{n}) \to 0$, it will show that $\lambda_{1}(\widetilde{\Omega_{n}})$ is also a minimizing sequence for the product $| \Omega | \lambda_{1}(\Omega)$.
  Since the number of possible picks is bounded by $N / 2$, this will prove that we can consider a minimizing sequence with bounded diameter.
  % todo until here

  We denote by $\eta_{n} = A_{n}B_{n}$ the width of the basis of the choice $(\eta_{n} \to 0)$ and $\omega_{n} = \Omega_{n} \cap B(\frac{A_{n} + B_{n}}{2}, 3\eta_{n})$.
  Let $\chi_{n}$ be a cut-off function which satisfies:
  \begin{enumerate}
    \item $\chi_{n} = 1$ outside $B(\frac{A_{n} + B_{n}}{2}, 3\eta_{n})$,
    \item $\chi_{n} = 0$ on the segment $A_{n}B_{n}$,
    \item $\chi_{n}$ is $C^{1}$ on $\overline{\widetilde{\Omega_{n}}}$,
    \item $\exists C > 0$ (independent of $n$) such that $| \nabla \chi_{n} | \leq \frac{C}{\eta_{n}}$.
  \end{enumerate}

  Let $u_{n}$ be the normalized first eigenfunction of $\Omega_{n}$.
  By construction $\chi_{n}u_{n} \in H_{0}^{1}(\widetilde{\Omega_{n}})$ as so it is admissible in the min formula that defines $\lambda_{1}$.

  Now, for any $C^{1}$ function $v$ we have
  \[
  | \nabla(v u_{n}) |^{2} = | u_{n} \nabla v + v \nabla u_{n}  |^{2} = u_{n}^{2} | \nabla v |^{2} + \nabla u_{n} \nabla (u_{n}v^{2})
  \] 
  or 
  \[
    | \nabla(v u_{n}) |^{2} = u_{n}^{2} | \nabla v |^2 + \text{div} (u_{n} v^2 \nabla u_{n}) + \lambda_{1}(\Omega_{n})u_{n}^2 v^2
  .\] 

  Replacing $v$ by $\chi_{n}$ and integrating on $\widetilde{\Omega_{n}}$ yields
  \[
    \int_{ \overline{\Omega_{n}}} \! | \nabla (\chi_{n}u_{n}) |^2 = \int_{ \overline{\Omega_{n}}} \! u_{n}^{2}| \nabla \chi_{n} |^2 + \lambda_{1}(\Omega_{n}) \int_{ \overline{\Omega_{n}}} \! \chi_{n}^2 u_{n}^2  
  .\] 
  Then, the variational definition of $\lambda_{1}(\widetilde{\Omega_{n}})$ is
  \[
  \lambda_{1}(\widetilde{\Omega}_{n}) \leq \lambda_{1}(\Omega_{n}) + \frac{\int_{ \overline{\Omega}_{n}} \! u_{n}^2 | \nabla \chi_{n} |^2 }{\int_{ \overline{\Omega}_{n}} \! \chi_{n}^2 u_{n}^2 }
  .\] 

  Now, using
  \todo[inline]{finish proof}
\end{proof}

\begin{theorem}
  Let $M \in \mathbb{N}$ and $\Omega$ be a polygon with $M$ edges.
  Then $\Omega$ cannot be a (local) minimum for $|\Omega| \lambda_{1}(\Omega)$ in the class $P_{M+1}$.
\end{theorem}

\begin{theorem}[Pólya]
  The equilateral triangle has the least first eigenvalue among all triangles of given area.
  The square has the least first eigenvalue among all quadrilaterals of given area.
\end{theorem}

\begin{proof}
  This proof is analogous to the Faber-Krahn Theorem, but uses Steiner Symmetrization instead of Schwarz rearrangement.
  We note that as the Steiner symmetrization shares the properties \ref{fk1} and \ref{fk2}, we know that any Steiner symmetrization will not increase the first eigenvalue.


  % converge a triangle to an equilateral by steiner symm wrt mediator
  We will construct a sequence of Steiner symmetrizations that makes a triangular domain converge to an equilateral triangle.
  Let $a_{n}$, $h_{n}$, and $A_{n}$ be the base, height, and one of the base's incident angles of the triangle $T_{n}$ that we obtain at step $n$.
  Then we have 
  \[
  \frac{h_{n}}{a_{n+1}} = \frac{h_{n+1}}{a_{n}} = \sin A_{n}
  .\] 
  Denote the ratio $x_{n} = \frac{h_{n}}{a_{n}}$.
  Then we have
  \[
  x_{n+1} = \frac{\sin^2 A_{n}}{x_{n}} = \frac{\sin^2(tan^{-1}(2x_{n}))}{x_{n}} = \frac{4x_{n}}{1 + 4x_{n}^2}
  .\] 
  Thus we have constructed the sequence $x_{n+1} = \frac{4x_{n}}{1 + 4x_{n}^2}$.
  This will converge to the fixed point of $f(x) = \frac{4x}{1 + 4x^2}$, which is $\frac{\sqrt{3}}{2}$.
  \begin{align*}
    \frac{4x}{1 + 4x^2} &= x \\
    x \left( 4x^2 - 3 \right ) &= 0 \\
  \end{align*}
  and so $x = \frac{\sqrt{3}}{2}$ is the fixed point of $f$.
    
  One can use elementary geometry to find that for an equilateral triangle with side length $a$, the height $h$ is  $\frac{\sqrt{3}}{2} a$.
  So $\frac{h}{a} = \frac{\sqrt{3}}{2}$, and thus our sequence converges to the value characteristic of equilateral triangles.
  Moreover, by Sverak's Theorem, the sequence of triangles $\gamma$-converges to the equilateral triangle which we will denote by $T_{e}$.
  Then, for an initial triangle domain $T$, we have shown 
  \[
  \lambda_{1}(T_{e}) = \lim \lambda_{1}(T_{n}) \leq \lambda_{1}(T)
  .\] 


  For quadrilaterals we can use a more elementary proof.
  One can show that by choosing a sequence of three Steiner symmetrizations you can transform any quadrilateral into a rectangle \cite{henrot}.
  \todo[inline]{Use Example eigenvalue def to show square minimizes eigen}.
\end{proof}



\begin{theorem}
  For $n \geq 3$ the regular polygon with $n$ sides is an extreme point for the first eigenvalue of the Dirichlet Laplace operator among polygons with $n$ sides and a fixed area.
\end{theorem}

\todo[inline]{pg 56 of bogosel paper}
\begin{proof}

  By \ref{eqmin}, our problem is equivalent (up to homothety) to solving the problem
  \[
  \min_{P \in P_{n}} \lambda_{1}(P) + |P|
  .\] 
  \todo[inline]{finish}

\end{proof}

\break



\break

\chapter{A Numerical Approach}
\break

\section{Overview}

\todo[inline]{Copy this section from Final Essay from last year}

\break

\section {Method of Fundamental Solutions}

We will use the method of fundamental solutions (MFS) to compute the eigenvalues of a given polygon.
The following construction is based on a similar method by Alves and Antunes \cite{fund}

Our goal is to numerically solve the Helmholtz equation with Dirichlet boundaries

\[
  \begin{cases}
    - \Delta u = \lambda u  & \text{in } \Omega \\
    u = 0 & \text{on } \partial \Omega
  \end{cases}
\] 
We will consider the group of functions which satisfy $- \Delta u = \lambda u$ that are of the form
\[
  u = a_1 \phi_{1}^\lambda + \ldots + a_{N} \phi_{N}^\lambda
,\] 

where $\phi_{i}^\lambda, i = 1,2,\ldots,M$ are fundamental radial solutions of $- \Delta u = \lambda u$ with singularities laying outside of $\Omega$.
Let $(y_{i})$ be the singularities of $\phi_{i}^\lambda$ outside of $\Omega$.

To find the coefficients $a_1,\ldots,a_{N}$ we impose the Dirichlet boundary condition on a discretization of $\partial \Omega$.
Let $(x_{i})$ be a discretization of $\partial \Omega$, and let $x_{N+1} \in \Omega$.
This leads to a system of equations 
\[
  \begin{cases}
    u(x_{i}) = 0 & \text{if } 1 \leq i \leq N \\
    u(x_{i}) = 1 & \text{if } i = N + 1 
  \end{cases}
\] 
Note that the equation when $i = N + 1$ is used to guarantee that $u(x) \not\equiv 0$ \cite{fund2}.

Obviously we are interested when the system has non-trivial solutions.
This occurs when the matrix $A_{\lambda} = (\phi_{i}^{\lambda}(x_{j}))_{i,j = 1}^N$ is singular.
As this shows the existence of an eigenfunction, we can find eigenvalues using the determinate of the matrix $A_{\lambda}$.
Specifically, we can find the eigenvalues of $\Omega$ on some interval $I$ by locating the values $\lambda \in I$ where $\det A_{\lambda} = 0$.
Once we have found an eigenvalue, we can solve the system to find a corresponding eigenfunction.


To apply MFS to our specific problem, we need to find suitable radial functions as well as $(x_{i})$ and $(y_{i})$.

First, we will find suitable radial functions.
Let $\phi := x(r)$ be a radial function in polar coordinates.
Then Helmholtz's equation becomes

  \begin{align*}
    -x'' - \frac{1}{r} x' &= \lambda x \\
    r^2 x'' + r x' + r^2 \lambda x &= 0.
  \end{align*}

Substituting in $s = \sqrt{\lambda} r$ we have

\[
  s^2 y'' + s y' + s^2 y = 0
,\] 

where $y(s) = x(r)$.
Note this is a specific case of Bessel's differential equation.
Thus, our radial fundamental solutions can be Bessel functions of order $0$.
We choose to use the Hankel function of the first kind with order $0$ as it is the most efficient computationally.

\begin{definition}
  We define the Bessel function of the first kind with order $0$ in the following way
  \[
    J_{0}(x) = \frac{1}{\pi} \int_{ 0}^{\pi} \! \cos(x \sin \tau) \, \mathrm{d}\tau 
  .\] 
  We define the Bessel function of the second kind with order $0$ in the following way

  \[
    Y_{0}(x) = \frac{4}{\pi^2} \int_{0}^{\frac{1}{2}\pi} \! \cos(x \cos \tau) \left( e + \ln \left( 2x \sin^2 \tau \right) \right) \, \mathrm{d}\tau 
  .\] 
  Finally, we define the Hankel function (of the first kind) with order $0$ as
  \[
  H_{0}(x) = J_{0}(x) + i Y_{0}(x)
  .\] 
\end{definition}

Thus our fundamental solutions will be of the form $\phi_{i}^\lambda = H_{0} ( \sqrt{\lambda} |x - y_{i} | ) $.


\todo[inline]{Should I go into how to choose $(x_{i}), (y_{i})$ or refer to various papers and summarize?}

\break

\section{Optimization}
In the previous section we outlined the method of fundamental solutions, a method to calculate the eigenvalues of our equation.
In this section we will outline a method to calculate the derivative of the eigenvalue and use gradient descent to find extremum.

From \ref{der}, the derivative of an eigenvalue is given by

\[
  \lambda_{k}'(0) := - \int_\Omega \! \left( \pderiv[1]{u}{n}  \right)^2 V.n \, \mathrm{d}\sigma 
.\] 

\todo[inline]{Maybe derive formula? IDK how involved it is.}

As we are taking the derivative with respect to the domain, we will begin by defining vector fields that allow us to write the derivative with respect to geometric parameters.
We will find particular vector fields $V$ which allow us to compute the derivative with respect to the coordinates of the vertices.
Fix a vertex and label it $v_{0}$.
Next, label the remaining vertices $v_{1}, v_2, \ldots, v_{N - 1}$ going around the polygon counterclockwise.
That is, for a vertex $v_{i}$ the adjacent vertices should be $v_{i-1},v_{i+1}$ modulo $N$.
Finally, take $(x_{i}, y_{i})$ to be the coordinates of $v_{i}$.

To find the derivative of $ \lambda_1$ with respect to $x_{i}$ we make a perturbation of $v_{i}$ with $(1,0)$.
This induces a perturbation of the adjacent edges of the boundary, which we will denote as $E_{i-1,i}$ and $E_{i,i+1}$.
For our particular case $V$ will have the following form on the boundary

\[
  \begin{cases}
    L_{i-1,i}(x,y) & (x,y) \in E_{i-1,i} \\
    L_{i,i+1}(x,y) & (x,y) \in E_{i,i+1} \\
    0   & \text{otherwise}
  \end{cases}
\] 

where $L_{j,k}  : E_{j,k} \to [0,1] $ is the following affine function 

\[
  L_{j,k}(x,y) =  
  \begin{cases}
    (x_{k} - x_{j})^{-1} (x - x_{i}) & \text{if } x_{i} \not = x_{i+1} \\
    0 & \text{otherwise} 
  \end{cases}
\] 

Denote the outer normal of the edge $E_{i,i+1}$ by $n_{j,j+1} = (n_{j,j+1}^1,n_{j,j+1}^2)$.
Then we can rewrite the derivative of the fundamental eigenvalue as

\[
  \D[1]{\lambda_1}{x_{i}} = - \int_{ E_{i-1,i}} \! L_{i-1,i} \left( \pderiv[1]{u}{n}  \right)^2 n_{i-1,i}^1  \, \mathrm{d}\sigma - \int_{ E_{i,i+1}} \! L_{i+1,i} \left( \pderiv[1]{u}{n}  \right)^2 n_{i,i+1}^1  \, \mathrm{d}\sigma
.\] 

Likewise, we can find the derivative with respect to the $y$ value as 
\[
  \D[1]{\lambda_1}{x_{2i}} = - \int_{ E_{i-1,i}} \! L_{i-1,i} \left( \pderiv[1]{u}{n}  \right)^2 n_{i-1,i}^2  \, \mathrm{d}\sigma - \int_{ E_{i,i+1}} \! L_{i+1,i} \left( \pderiv[1]{u}{n}  \right)^2 n_{i,i+1}^2  \, \mathrm{d}\sigma
.\] 

Following these computations, we have all of the pieces needed to optimize the fundamental eigenvalue using gradient descent.

\todo[inline]{Give Numerical Results}

\todo[inline]{Talk about why we're allowed to minimize sum of area and eigenvalue instead of forcing the area constraint}

\break

 
 
\printbibliography
 
\end{document}
