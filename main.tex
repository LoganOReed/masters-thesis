\documentclass[12pt]{report}

\usepackage[a4paper,bindingoffset=0.2in,%
            left=1.5in,right=1in,top=1in,bottom=1in,%
            footskip=.25in]{geometry}


\usepackage[utf8]{inputenc}
\usepackage[english]{babel}
\pagenumbering{roman}
\usepackage{graphicx} 
\usepackage{biblatex}
\usepackage{csquotes} % Biblatex throws warning if not included
\usepackage{amsmath}
\usepackage{amsthm}
\usepackage{setspace} 
\usepackage{amssymb} 
\usepackage{esint} 
\usepackage{cool} % A ridiculous number of macros for complex symbols

% specific external file paths
\addbibresource{refs.bib}
\graphicspath{ {./graphics/} }



\newtheorem{theorem}{Theorem}[section]
%\def\thetheorem{\unskip}
\newtheorem{proposition}[theorem]{Proposition}
%\def\theproposition{\unskip}
\newtheorem{conjecture}[theorem]{Conjecture}
\def\theconjecture{\unskip}
\newtheorem{corollary}[theorem]{Corollary}
\newtheorem{lemma}[theorem]{Lemma}
\newtheorem{observation}[theorem]{Observation}
%\def\thelemma{\unskip}
\newtheorem{definition}{Definition}
\numberwithin{definition}{section}
%\def\thedefinition{\unskip}
\newtheorem{remark}{Remark}
\def\theremark{\unskip}
\newtheorem{question}{Question}
\def\thequestion{\unskip}
\newtheorem{example}{Example}
\def\theexample{\unskip}
\newtheorem{problem}{Problem}
\newtheorem{exercise}[theorem]{Exercise}

\begin{document}

  \doublespacing
  
\begin{titlepage}

   \begin{center}
      

      The Pólya–Szegő Conjecture on Polygons: A Numerical Approach

            
     
       By
              


      LOGAN REED

     
            
        A thesis submitted to the \\
       Graduate School–Camden\\
       Rutgers, The State University of New Jersey\\
       In partial fulfullment of the requirements\\
       For the degree of Master of Science\\
       Graduate Program in Mathematical Sciences \\
       Written under the direction of \\
       Siqi Fu\\
       And approved by \\
       \noindent\rule{4cm}{0.4pt}\\
       Dr. One\\
      \noindent\rule{4cm}{0.4pt}\\
       Dr. Two\\
           \noindent\rule{4cm}{0.4pt}\\
       Dr. Three\\
                \noindent\rule{4cm}{0.4pt}\\
       Dr. Four\\
         \vspace{0.8cm}
       Camden, New Jersey\\
       May 2023
            
       \vspace{0.8cm}
     
  
        
            
   \end{center}
   
\end{titlepage}




\break
TEST PAGE

  \begin{center}

\break
  THESIS ABSTRACT\\
 
    TODO SOMETHING ABOUT THE ABSTRACT THAT IS ABOUT THIS LONG OR SO \\
         by LOGAN REED\\
     
     Thesis Director: \\
     Siqi Fu
  \end {center}

  The topic that I chose to explore for this thesis is a study of the eigenvalues of the Dirichlet Laplacian on a two dimensional domain and \ldots
  
	% The topic that I chose to explore for this thesis is a study of the eigenvalues of the Dirichlet Laplacian on a two dimensional domain and the properties that arise as a direct consequence to them. The eigenvalues of a given domain produce so many surprising insights that simply the study of a triangular domain has many directions in which one could explore. What I love about this topic is that throughout my study, a resource from 1966 could take me to a resource from the 1700's which could lead me all the way back to 2013. It is a topic rich with exploration, both new and old which really gave me a good picture of what modern research in pure mathematics can look like.
	% 
	% The goal with this thesis was simple; understand a few of the implications brought up by the famous question of Mark Kac, ``Can one hear the shape of a drum?" and then possibly, with enough effort, make some type of original contribution to the topic. 
	% 
	% In my study of these domains, I was inevitably brought back to the isoperimetric inequality leading me to a process called Steiner Symmetrization. This process was introduced to me by \cite{Polya}.  We were able to use some basic trigonometry to fill in the detail in the treatise of George Polya and come up with simplified proofs of special cases of such symmetrization. 
	%  
\break



\begin{center}
Acknowledgment
\end{center}
    I would be remiss if I did not take a moment to express appreciation for all of the people who have helped me through this process.
    % I would be remiss if I did not take a moment to express appreciation for all of the people who have helped me through this process. 
    % 
    % Dr. Siqi Fu is a person I must thank not only for supervising my research throughout the writing of this thesis, but also helping me make the decision all those years ago to pursue both bachelor and masters degrees in mathematics and later encouraging me to continue my education in pursuit of a PhD as well. Without his guidance and encouragement at multiple steps throughout my academic career, my confidence to pursue a path of learning mathematics may have wavered or worse, not even have began. 
    % 
    % As a person that has spent my entire secondary educational career at Rutgers University–Camden, I also must thank the rest of the Department of Mathematics. At a smaller school like Rutgers I was given the unique opportunity to develop closer relationships with my professors. Because of this, at every step along the way, through each class, conversation and piece of advice, the professors in the math department at Rutgers each hold an integral role in my development as a student of mathematics. Through their mathematical and pedagogical prowess, I feel as though the Rutgers professors have prepared me to continue my education and hopefully someday make a real contribution to the field of Mathematics. For this education and guidance I am truly grateful. 
    % 
    % Lastly, I would like to thank all of my family and friends. Without the love and support of my parents, I would not have been able to pursue all of the opportunities I have been afforded at Rutgers. I would also like to thank Rachel and Dixon for encouraging me at every step along the way thus far (and spell checking this document for me). 
    





	
\break

\tableofcontents


\newcommand{\comment}[1]{}


\break


\pagenumbering{arabic}
\pagestyle{myheadings}



\chapter{Introduction}

\break
\section {Physical Motivation}


Physical drums consist of a rigid shell with a membrane which produces sound when hit.
A similar thing can be "created" in a pure mathematical setting by studing specific partial differential equations over a closed region.
Specifically, the frequencies of the drum membrane corresponds to the eigenvalues of the Dirichlet Laplacian.
This construction makes it possible to "hear" drums where the shape of the drumhead is any closed and simple curve.
To do this, we begin with the shape of our drumhead, which is a region of the real plane bounded by piecewise smooth curves.
Since we wish to emulate the physical properties of a drum, we want to define some system that models the vibration of the drum membrane which produces the sound.
This is done using the wave equation over our boundary, with the boundary condition that the function is 0 on the boundary.
The Dirichlet Laplacian allows us to model the physical vibration of the drum, and we can calculate its eigenvalues to find the fundamental frequency and overtones.
Thus, by modeling the vibration of the drumhead using the Dirichlet Laplacian and studying its properties, we can produce "sound" via the eigenvalues.

In 1877, Lord Rayleigh conjectured the following \cite{rayleigh}

\begin{quote}
If the area of a membrane be given, there must evidently be some
form of boundary for which the pitch (of the principal tone) is the
gravest possible, and this form can be no other than the circle.
\end{quote}

In 1923, Faber published a proof which was followed by an independent proof by Krahn in 1925 \cite{krahn}.
\begin{theorem}[Faber-Krahn]
 Let $c$ be a positive number and $B$ the ball with volume $c$. Then,
 \[
   \lambda_{1}(B) = \min \left\{ \lambda_{1}(\Omega), \Omega \text{ open subset of } \mathbb{R}^{N}, |\Omega| = c \right\} 
 .\] 
\end{theorem}
From the Faber-Krahn inequality, we know that for any drumhead with a given area, the circle is the one with the lowest tone.
In 1951, Polya and Szego conjectured that a similar statement holds for drumheads with a polygonal shape \cite{henrot}.
This conjecture has been shown to be true for 3 and 4 sided polygons, but remains unproven for any other number of sides.

There are two main hurdles that are halting progress on this conjecture.
The first is that the tools that were used to prove both Lord Rayleighs conjecture as well as the small cases for the Polya-Szego conjecture are not available when the number of sides is greater than four.
The main tool that is used is called Steiner Symmetrization, and when there are more than four sides this symmetrization method creates additional sides at each step.

The purpose of this paper is to show a specific method for running numerical approximations to suggest that this conjecture is indeed true.
This is done using a method based on fundamental solutions \cite{code}.
Specifically, we consider all functions that satisfy the Laplace Equation and then solve for the linear coefficients using the boundary conditions.
Once this is done we use gradient descent to find the polygon with the minimum first eigenvalue, which is equivalent to the first fundamental tone of the drum.



% Page 66 in Evans PDE
% Page 20 in Evans is better

Consider a homogeneous elastic drumhead, or membrane, stretched over a rigid frame.
We will represent the frame as a domain $\Omega \subset \mathbb{R}^{2}$.
Take the function $u(x,y,t)$ to be the vertical displacement of the membrane from its resting position.
Then for any disk $D \subset \Omega$, Newton's second law of motion states that

$$ \int_{\partial D} T \frac{\partial u}{\partial \textbf{n}} \, dS = \int_{ D} \rho u_{tt} \, dA $$

where T is the constant tension, $\rho$ is the density constant, and $\textbf{n}$ is the outward normal of the boundary.
By the divergence theorem, we have

$$ \int_{D} T \Delta u \, dA = \int_{D} \rho u_{tt} \, dA $$

where $\Delta$ is the Laplace operator. From this we can get the wave equation on $\Omega$

$$ u_{tt} = c^{2} \Delta u $$

where we define u to be 0 on the boundary and where $c = \sqrt{T / \rho}$.
We can solve this wave equation using $u(x,y,t) = T(t)V(x,y)$ which gives us

$$ \frac{T''}{c^{2}T} = \frac{\Delta V}{V} = - \lambda $$

and finally we have reduced our problem to the Dirichlet Laplacian

$$ \Delta V = - \lambda V $$

where $V$ on the boundary is zero.

NOTE: The best reference I could find is Logan's Applied Partial Differential Equations.
I could also use Ryans paper

In the next section, we will start from the Dirichlet Laplacian and introduce the conjectures in a formal setting.

\break

\section {Polya-Szego's Conjecture}


\begin{enumerate}
  \item Introduce Rigorous Definitions from 1.1.2 Henrot
  \item Dirichlet Laplacian eigenvalues prereqs
  \item Faber Krahn
  \item Polya-Szego Conjecture \cite{isoperimetric}
\end{enumerate}

\break

\section{Known Results}

\begin{enumerate}
  \item All Explicit Cases
  \item Tools for n=3 and n=4
\end{enumerate}

\break

\section {Numerical Analysis Tools}
\break

\chapter {Background}

\break

\section{Notations and Definitions}

\break

\section{Measure Theory}

\begin{definition}
  A Measure space is a pair $(X,A)$ where $X$ is a non-empty set and $A$ is a $\sigma$-algebra of subsets of $X$.
  That is, $A$ satisfies the following
  \begin{enumerate}
    \item $\emptyset \in A$
    \item For countably many $A_{j} \in A$, $\cup A_{j} \in A$ 
    \item If $B \in A$, then $X - B \in A$.
  \end{enumerate}
\end{definition}

  TODO Integration and maybe supports
\break

\section{Function Spaces}

\begin{definition}
A complex linear space $\mathbb{H}$ is called a \textit{normed linear space} if there exists a map $|| \cdot || : \mathbb{H} \to \mathbb{R}^{+} $ such that for any $x, y \in \mathbb{H}$ and $ \lambda \in \mathbb{C}$,

\begin{enumerate}
  \item $|| \lambda x || = |\lambda| || x ||$
  \item $|| x + y || \leq || x || + || y ||  $
  \item $|| x || \geq 0$, and $|| x || = 0$ if and only if $x = 0$
\end{enumerate}

\end{definition}

\begin{definition}
  A complex linear space $\mathbb{H}$ is called an inner product space with inner product $ \langle \cdot, \cdot \rangle  : \mathbb{H} \times \mathbb{H} \to \mathbb{C} $ if for any $x,y,z \in \mathbb{H}$ and $\lambda \in \mathbb{C}$,
  \begin{enumerate}
    \item $ \langle \lambda x, y \rangle = \lambda \langle x, y \rangle  $
    \item $ \langle x,y \rangle = \overline{ \langle y,x \rangle }$
    \item $ \langle x + y, z \rangle = \langle x,z \rangle + \langle y,z \rangle $
    \item $ \langle x,x \rangle \geq 0$, and $ \langle x,x \rangle = 0 $ if and only if $x = 0$.
  \end{enumerate}
\end{definition}

\begin{definition}
  A Hilbert space is a complete inner product space
\end{definition}
\break

\section{PDEs}

\begin{definition}[Laplacian]
  \[
  - \Delta u := - \sum_{i=1}^N \pderiv[2]{u}{x_{i}} 
  .\] 
\end{definition}

\begin{definition}
  The Dirichlet Laplacian is the Laplace Operator subject to Dirichlet boundary conditions.
  That is, we call $u$ a solution to the Dirichlet Laplacian if $u$ is a solution to

  \[
  \begin{cases}
    \Delta u = \lambda u & \text{in } \Omega \\
    \,\,\,\,\, u = 0  & \text{on } \partial \Omega \\
  \end{cases}
  .\] 
\end{definition}
\break

\section{Calculus of Variations}

\break

\section{Tools}

\begin{definition}[Schwarz Rearrangement] \label{schwarz}
  For any measurable set $\omega$ in $\mathbb{R}^{N} $, we denote by $\omega^{*}$ the ball of same volume as $\omega$.
  If $u$ is a non-negative measurable function defined on a measurable set $\Omega$ and vanishing on its boundary $\partial \Omega$,
  we denote by $\Omega(c) = \{ x \in \Omega \,|\, u(x) \geq c \}$ its level sets.
  The Schwarz rearrangement of $u$ is the function $u^{*}$ defined on $\Omega^{*}$ by 
  \[
    u^{*}(x) = \sup\{c / x \in \Omega(c)^{*}\}
  .\] 
  
\end{definition}

Without loss of generality, we fix the hyperplane of symmetry to be $x_{N} = 0$.
Let $N \geq 2$ and $\Omega \subset \mathbb{R}^{N} $ be a measurable set.
We denote by $\Omega'$ the projection of $\Omega$ on $\mathbb{R}^{N-1} $, and for every $x' \in \mathbb{R}^{N-1} $ we denote by $\Omega(x')$ the projection of $\Omega$ with $ \left\{ x' \right\} \times \mathbb{R}^{}  $.

\begin{definition}[Steiner Symmetrization] \label{steiner}
Let $\Omega \subset \mathbb{R}^{N} $ be measurable.
Then the set 
\[
\Omega^{*} := \left\{ x = (x', x_{N}) : - \frac{1}{2} |\Omega(x')| < x_{N} < \frac{1}{2} |\Omega(x')|, x' \in \Omega' \right\} 
\] 
is the Steiner symmetrization of $\Omega$ with respect to the hyperplane $x_{N}$ = 0.
\end{definition}

\begin{theorem} \label {fk1}
 Let $\Omega$ be a measurable set and $u$ be a non-negative measurable function defined on $\Omega$ and vanishing on its boundary $\partial  \Omega$. 
 Let $\phi$ be any measurable function defined on $\mathbb{R}^{+} $ with values in $\mathbb{R}^{} $, then
 \[
   \int_\Omega \! \phi(u(x)) \, \mathrm{d}x = \int_{\Omega^*} \! \phi(u^*(x)) \, \mathrm{d}x  
 .\] 
\end{theorem}

\begin{theorem}[Pólya's Inequality] \label{fk2}
  Let $\Omega$ be an open set and $u$ a non-negative function belonging to the Sobolev space $H_{0}^{1}(\Omega)$.
  Then $u^{*} \in H_{0}^{1}(\Omega^{*})$ and 
  \[
    \int_\Omega \! | \nabla u(x)  |^2  \, \mathrm{d}x \geq \int_{\Omega^{*}} \! | \nabla u^{*}(x) |^2 \, \mathrm{d}x 
  .\] 
\end{theorem}


TODO Steiner Symmetrization
\break

\chapter{Eigenvalues of the Dirichlet Laplacian}
\break

\section{Definition}


\begin{definition}[Rayleigh Quotient] \label{rq}
 For an operator $L$, we define the Rayleigh quotient to be 
 \[
   R_{L}[v] := \frac{\sum_{i,j=1}^N \int_{ \Omega} \! a_{ij}(x) \pderiv[1]{v}{x_{i}} \pderiv[1]{v}{x_{j}}  \, \mathrm{d}x + \int_{ \Omega} \! a_0(x)v^2(x) \, \mathrm{d}x }{\int_{ \Omega} \! v(x)^2 \, \mathrm{d}x }
 .\] 

\end{definition}

This is used to express the first eigenvalue of the Dirichlet Laplacian in the following way

\[
  \lambda_{1}(\Omega) = \min_{v \in H_{0}^{1}(\Omega), v \not = 0 } \frac{\int_{ \Omega} \! | \nabla v(x) | ^2 \, \mathrm{d}x }{\int_{ \Omega} \! v(x)^2 \, \mathrm{d}x }
.\] 

\begin{theorem}
    Let $\Omega$ be a bounded open set. We assume that $\lambda_{k}'(\Omega)$ is simple.
  Then, the functions $t \to \lambda_{k}(t), t \to u_{t} \in L^2(\mathbb{R}^{N} )$ are differentiable at $t = 0$ with
  \[
    \lambda_{k}'(0) := - \int_\Omega \! \mathrm{div} (|\nabla u|^2 V) \, \mathrm{d}x 
  .\] 
  If, moreover, $\Omega$ is of class $C^2$ or if $\Omega$ is convex, then
   \[
    \lambda_{k}'(0) := - \int_\Omega \! \left( \pderiv[1]{u}{n}  \right)^2 V.n \, \mathrm{d}\sigma 
  \] 
  and the derivative $u'$ of $u_{t}$ is the solution of

\[ 
  \begin{cases}
    - \Delta u' = \lambda_{k}u' + \lambda_{k}'u & \mathrm{in }  \Omega \\
    \phantom{--}u'  = - \pderiv[1]{u}{n} V.n & \mathrm{on } \partial \Omega \\
    \phantom{-}\int_\Omega \! u u' \, \mathrm{d} \sigma = 0 .
  \end{cases}
\] 

\end{theorem}
\break

\section{Known Results}

\begin{enumerate}
  \item invariant under translations rotations
  \item homothety
  \item continuous
\end{enumerate}


% Faber Krahn Tools


\begin{theorem}[Faber-Krahn] \label{fk}
 Let $c$ be a positive number and $B$ the ball with volume $c$. Then,
 \[
   \lambda_{1}(B) = \min \left\{ \lambda_{1}(\Omega), \Omega \text{ open subset of } \mathbb{R}^{N}, |\Omega| = c \right\} 
 .\] 
\end{theorem}

\begin{proof}
  This proof is a straightforward application of Schwarz rearrangement (\ref{schwarz}) \cite{henrot}.
  Let $\Omega$ be a bounded open set of measure $c$ and $\Omega^* = B$ be the ball of the same volume.
  Let $u_1$ be a en eigenfunction with eigenvalue $\lambda_{1}(\Omega)$ and $u_1^*$ its Schwarz rearrangement.
  Using \ref{fk1} we have
  \[
   \int_{\Omega^{*}} \! u_1^{*}(x)^2 \, \mathrm{d}x = \int_{ \Omega} \! u_1(x)^2 \, \mathrm{d}x
  .\] 
  Further, using \ref{fk2} we have
  \[
     \int_{\Omega^{*}} \! | \nabla u_1^{*}(x) |^2 \, \mathrm{d}x \leq \int_\Omega \! | \nabla u_1(x)  |^2  \, \mathrm{d}x
  .\] 

  Using Rayleigh quotients (\ref{rq}) we get the following

  \[
    \lambda_{1}(\Omega^{*}) \leq \frac{\int_{\Omega} \! | \nabla u_1^{*}(x) |^2 \, \mathrm{d}x }{\int_{\Omega} \! u_1^{*}(x)^2 \, \mathrm{d}x }
  .\] 

  \[
    \lambda_{1}(\Omega) = \frac{\int_{ \Omega} \! | \nabla u_1(x) | ^2 \, \mathrm{d}x }{\int_{ \Omega} \! u_1(x)^2 \, \mathrm{d}x}
  .\] 

  Using the previous two statements yields the desired results.
\end{proof}

\break

\section{Polygons}

Note $P_{N}$ is the class of plane polygons with at most N edges.

\begin{theorem}
  Let $a > 0$ and $N \in \mathbb{N}$ be fixed.
  Then the problem
  \[
    \min \left\{ \lambda_{1}(\Omega), \Omega \in P_{N}, |\Omega| = a \right\} 
  \] 
  has a solution.
\end{theorem}

\begin{proof}
  47 henrot
\end{proof}

\begin{theorem}
  Let $M \in \mathbb{N}$ and $\Omega$ be a polygon with $M$ edges.
  Then $\Omega$ cannot be a (local) minimum for $|\Omega| \lambda_{1}(\Omega)$ in the class $P_{M+1}$.
\end{theorem}

\begin{theorem}[Pólya]
  The equilateral triangle has the least first eigenvalue among all triangles of given area.
  The square has the least first eigenvalue among all quadrilaterals of given area.
\end{theorem}

\begin{proof}
  pg 50 henrot
\end{proof}

TODO: Explain Issue with proof method for cases N >= 5

\break





\printbibliography

\chapter{A Numerical Approach}
\break

\section{Overview}
\begin{enumerate}
  \item Method of fundamental solutions using hankel functions
  \item Additional requirements for the matrix for the next part
  \item Calculate derivative of first eigenvalue using boundary method
  \item Apply gradient descent to find minimum
\end{enumerate}
\break

\section {Method of Fundamental Solutions}
\begin{theorem}
  Laplace's Equation is invariant under rotations.
\end{theorem}

TODO: Define Hankel Functions and Show they are fundamental solutions to Laplace Eq

\break

\section{Alves and Antunes Stuff}
\break

% IDK if this is necessary
\section{Weak Formulations}
\break

% IDK if this is necessary either
\section{Finite Element Methods}
TODO Use Atkinson Book alongside chapter 6 of Evans PDE book

 
 
 
\end{document}
